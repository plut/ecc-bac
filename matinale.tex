\documentclass[10pt,professionalfont]{beamer}
\usepackage[utf8]{inputenc}
\usepackage[T1]{fontenc}
\usepackage[francais]{babel}
\usepackage{unicode}
\usepackage{booktabs}
\usepackage{lmodern,sfmath}

\makeatletter
\usetheme{CambridgeUS}%<<<
\definecolor{bleu}{rgb}{.1,.4,.6}%1a6699
\definecolor{rouge}{rgb}{.6,.1,.15}%991a26

\mode<presentation>{
\setbeamercolor{structure}{fg=rouge!80}
\setbeamercolor{palette primary}{fg=white,bg=rouge}
\setbeamercolor{palette secondary}{fg=white,bg=rouge!90}
\setbeamercolor{palette tertiary}{fg=white,bg=rouge!80}
\setbeamercolor{palette quaternary}{fg=white,bg=rouge!70}
\setbeamercolor{title}{fg=white,bg=bleu!60}
\setbeamercolor{section in head/foot}{fg=white,bg=bleu!90}
\setbeamercolor{subsection in head/foot}{fg=white,bg=bleu}
\setbeamercolor{frametitle}{fg=white,bg=bleu!80}
\setbeamercolor{titlelike}{parent=palette quaternary}
\setbeamercolor{block title}{fg=white,bg=bleu!80}
\setbeamercolor{block body}{fg=black,bg=bleu!20}
% couleurs perso pour ce qui suit
\setbeamercolor{strong}{fg=rouge}
\setbeamercolor{emph}{fg=bleu}
\setbeamercolor{table head}{fg=white,bg=bleu!80}
\setbeamercolor{table odd}{bg=bleu!20}
\setbeamercolor{table even}{bg=bleu!10}
}
\useinnertheme{rectangles}
\setbeamertemplate{blocks}[default]
\setbeamertemplate{title page}[default][rounded=false,shadow=false]
%>>>
\def\strong#1{{\usebeamercolor{strong}\textcolor{fg}{\textbf{#1}}}}
\def\imp#1{{\usebeamercolor{emph}\textcolor{fg}{\emph{#1}}}}
% Section frames%<<<
\let\oldsection\section
\newenvironment<>{sectionblock}[1]{%
  \begin{actionenv}#2\def\insertblocktitle{#1}%
  \mode<presentation>{%
    \setbeamercolor{block body}{fg=white,bg=rouge!90}
  }\usebeamertemplate{block begin}}%
{\par\usebeamertemplate{block end}\end{actionenv}}%
\newenvironment<>{subsectionblock}[1]{%
  \begin{actionenv}#2\def\insertblocktitle{#1}%
    \mode<presentation>{%
            \setbeamercolor{block body}{fg=white,bg=rouge!80}
    }\usebeamertemplate{block begin}}%
{\par\usebeamertemplate{block end}\end{actionenv}}%

\newif\if@sectionframe@cnt \@sectionframe@cnttrue
\def\sectionframe{\@ifstar
  {\@sectionframe@cntfalse\@sectionframe}%
  {\@sectionframe@cnttrue \@sectionframe}%
}
\def\thesection{\@Roman\c@section}
\def\thesubsection{\@arabic\c@subsection}
\def\@sectionframe#1{%
  \def\insertsectionhead{}\def\insertsubsectionhead{}\begin{frame}%
  \begin{center}\begin{sectionblock}{}\hfil\Large\strut
  \if@sectionframe@cnt \stepcounter{section}\thesection~--~\fi
  #1\end{sectionblock}
  \addtocounter{section}\m@ne
  \end{center}\end{frame}\section{#1}}
\def\subsectionframe#1{%
  \def\insertsubsectionhead{}\begin{frame}\begin{center}
  \begin{sectionblock}{}\hfil\Large\strut\insertsectionhead
  \end{sectionblock}
  \begin{subsectionblock}{}%
  \stepcounter{subsection}
  \hfil \large \thesubsection. #1\end{subsectionblock}
  \addtocounter{subsection}\m@ne
\end{center}\end{frame}\subsection{#1}}
%>>>
% Autoriser les blocs sans titre%<<<
\setbeamertemplate{block begin}{\par\medskip
 \ifx\insertblocktitle\empty\else
 \par\vskip\medskipamount%
  \begin{beamercolorbox}[colsep*=.75ex]{block title}
    \usebeamerfont*{block title}\insertblocktitle%
  \end{beamercolorbox}%
 {\parskip0pt\par}%
  \ifbeamercolorempty[bg]{block title}
  {}
  {\ifbeamercolorempty[bg]{block body}{}{\nointerlineskip\vskip-0.5pt}}%
 \fi
  \usebeamerfont{block body}%
  \begin{beamercolorbox}[colsep*=.75ex,vmode]{block body}%
    \ifbeamercolorempty[bg]{block body}{\vskip-.25ex}{\vskip-.75ex}\vbox{}%
}%>>>
\RequirePackage{colortbl}
% Playing with colortbl%<<<
% definining \tablecolor: the default color for a table
\let\orig@CT@setup\CT@setup
\let\CT@tablecolor\@empty
\let\CT@tablefg\@empty
\let\CT@rowfg\@empty
%
\def\tablecolor#1{\def\CT@tablecolor{#1}}
% defining \tablecoloralt: alternating colors for table rows
\let\CT@tablecoloraltI\@empty
\let\CT@tablecoloraltII\@empty
\def\tablecoloralt#1#2{\def\CT@tablecoloraltI{#1}\def\CT@tablecoloraltII{#2}}
\def\tablefg#1{\noalign{\gdef\CT@tablefg{#1}}}
\def\rowfg#1{\noalign{\gdef\CT@rowfg{#1}}}
% modifying start/save macros%<<<
\def\CT@start{%
  \let\CT@arc@save\CT@arc@
  \let\CT@drsc@save\CT@drsc@
  \let\CT@row@color@save\CT@row@color
  \let\CT@cell@color@save\CT@cell@color
  \let\CT@tablecoloraltI@save\CT@tablecoloraltI
  \let\CT@tablecoloraltII@save\CT@tablecoloraltII
  \def\CT@tablerowsave{\the\tablerow}\tablerow 0
  \def\+{\noalign{\CT@samerow}}%
  \global\let\CT@cell@color\relax}
\def\CT@end{%
  \global\let\CT@arc@\CT@arc@save
  \global\let\CT@drsc@\CT@drsc@save
  \global\let\CT@row@color\CT@row@color@save
  \global\let\CT@cell@color\CT@cell@color@save
  \let\CT@tablecoloraltI\CT@tablecoloraltI@save
  \let\CT@tablecoloraltII\CT@tablecoloraltII@save
  \expandafter \tablerow \CT@tablerowsave
}%>>>
\newcount\tablerow \def\CT@samerow{\global\advance\tablerow -1}
\CT@everycr{\noalign{\global\let\CT@row@color\relax
  \global\let\CT@rowfg\@empty
  \global\advance\tablerow 1}\the\everycr}
\def\CT@setup{\orig@CT@setup
  \ifx\CT@tablecoloraltI\@empty
    \ifx\CT@tablecolor\@empty\else\CT@color{\CT@tablecolor}\fi
  \else \ifodd\tablerow \CT@color{\CT@tablecoloraltI}%
    \else \CT@color{\CT@tablecoloraltII}\fi\fi
  \ifx\CT@rowfg\@empty \else
    \global\setbox\z@ \hbox{{\color{\CT@rowfg}\unhbox \z@}}\fi
  \ifx\CT@tablefg\@empty \else
    \global\setbox\z@ \hbox{{\color{\CT@tablefg}\unhbox \z@}}\fi
} % >>>
% Tableaux (utilise colortbl)%<<<
\newenvironment<>{tableau}[1]{\begin{actionenv}%
  \usebeamercolor{table head}
  \usebeamercolor{table odd}\usebeamercolor{table even}
  \def\entete{\+\rowcolor{table head.bg}\rowfg{table head.fg}}%
  \def\arraystretch{1.2}
  \tablecoloralt{table odd.bg}{table even.bg}\begin{tabular}{#1}}
  {\end{tabular}\end{actionenv}}
%>>>
% Macros perso
\def\abs#1{\left|#1\right|}
\def\pa#1{\left(#1\right)}
\def\bib#1{{\usebeamercolor{emph}\textcolor{fg}{~[#1]}}}
\def\F{\mathbb{F}}
\def\prob{\mathcal{P}}

\makeatother

\begin{document}
\title[Choix des courbes elliptiques]{Diversité et transparence : choix des courbes elliptiques}
\author[J.-P. Flori, J. Plût]{Jean-Pierre Flori, Jérôme Plût, Jean-René
Reinhard, Martin Ekerå}
\institute[ANSSI]{ANSSI/SDE/ST/LCR}

\begin{frame}<handout:0> \titlepage
\end{frame}
\sectionframe{Introduction}
\begin{frame}\frametitle{Les courbes elliptiques en cryptographie}
\begin{itemize}
\item Proposées en 1985 par Koblitz et Miller.
\item Fournissent un groupe abélien fini
où le logarithme discret est \strong{difficile}
(plus que dans les groupes multiplicatifs).
\item Standardisées à partir de 2000:
\begin{center}\begin{tableau}{lll}
\entete Année & Courbes & Tailles \\
2000 & NIST & 192, 224, 256, 384, 521\\
2005 & Brainpool & 160, 192, 224, 256, 320, 384, 512\\
2010 & OSCCA & 256 \\
2011 & ANSSI & 256 \\
\end{tableau}\end{center}
\item Plus quelques propositions académiques.
\item En pratique, on trouve surtout dans la nature les courbes NIST.
\end{itemize}
\end{frame}
\begin{frame}\frametitle{Pourquoi les courbes elliptiques ?}
\begin{itemize}
\item Un outil classique en cryptographie est
l'échange de clés de Diffie-Hellman, qui repose sur la relation
\[ (g^a)^b = g^{ab} = (g^b)^a, \]
valable dans le groupe multiplicatif des entiers modulo $p$.
\item La sécurité repose sur le problème du \strong{logarithme discret}:
étant donnés $g$ et~$g^a$, il est difficile de calculer $a$.
\item Ce groupe multiplicatif est cependant vulnérable
à certaines attaques (« crible algébrique »).
\item Solution : augmenter la taille de $p$...
\item ... ou alors, remplacer le groupe multiplicatif
par un groupe résistant à ces attaques.
\end{itemize}
\end{frame}
\begin{frame}\frametitle{Le groupe des points d'une courbe elliptique}
\begin{itemize}
\item Une courbe elliptique est donnée par l'équation
\[ y^2 = x^3 + a x + b \pmod{p}, \]
où $p$ est un nombre premier ($≠ 2,3$) et $a, b$ sont deux paramètres.
\item Les points de la courbe forment un groupe abélien
(noté additivement).
\item Le cardinal $N$ de ce groupe est environ $p$; en fait,
\[ \abs{N-(p+1)} ≤ 2√p. \]
\end{itemize}
\begin{block}{}
En général, le groupe des points d'une courbe elliptique
se comporte comme un « groupe générique » :
le logarithme discret a une complexité exponentielle.
\end{block}
Il est donc possible d'atteindre une sécurité de 128 bits
avec une taille de clé de 256 bits.
\end{frame}
\begin{frame}\frametitle{Pourquoi standardiser ?}
\begin{block}{}
\strong{En général,} le groupe des points d'une courbe elliptique
se comporte comme un « groupe générique » :
le logarithme discret a une complexité \strong{exponentielle}.
\end{block}
\begin{itemize}
\item Plus précisément, la complexité du logarithme discret
est dominée par $√{q}$, où $q$ est
le \imp{plus grand diviseur premier} du nombre de points de la courbe.
\begin{itemize}
\item Solution : avoir un nombre de points (presque) premier.
\end{itemize}
\item Certaines courbes particulières sont plus vulnérables :
le problème du logarithme discret peut être transféré
dans un groupe plus facile.
\begin{itemize}
\item Solution : éviter ces cas particuliers.
\end{itemize}
\end{itemize}

\begin{block}{}
Ces solutions sont gourmandes en calculs,
il n'est donc pas réaliste d'envisager
fabriquer une nouvelle courbe à la volée pour chaque échange.
\end{block}
\end{frame}
\begin{frame}\frametitle{Deuxième phase de standardisation}
Les premières courbes standardisées ont été produites
au début des années 2000, c'est-à-dire
une fois la recherche dans le domaine suffisamment avancée :
\begin{itemize}
\item possibilité de produire des courbes cryptographiques
(Schoof, Elkies, Atkin) ;
\item identification des classes de courbes faibles.
\end{itemize}
\begin{block}{}
À notre connaissance, ces courbes sont toujours sûres.
\end{block}
Mais de nouvelles préoccupations sont apparues depuis :
\begin{itemize}
\item doutes sur le processus de génération
(possibilité de publier une courbe secrètement vulnérable ?) ;
\item émergence des attaques latérales
(\emph{« side-channel attacks »}) ;
\item progrès scientifiques dans des domaines proches
(logarithme discret multiplicatif...).
\end{itemize}
\begin{block}{Juin 2015}
Le NIST organise un atelier sur le thème de
la standardisation des courbes elliptiques.
\end{block}
\end{frame}
\sectionframe{Sécurité}
\begin{frame}\frametitle{Aspects de la sécurité d'une courbe}
Qu'est-ce que qu'une « bonne » courbe pour la cryptographie ?
\begin{itemize}
\item Le problème du logarithme discret est difficile.
\item Les implémentations de la courbes sont résistantes
(par exemple aux attaques par canaux auxiliaires).
\item La courbe ne présente aucun signe particulier suspect.
\item Il est possible de réaliser des implémentations optimisées.
\item La courbe possède des propriétés particulières intéressantes.
\end{itemize}
\begin{block}{Conditions incompatibles}
Certaines de ces conditions sont incompatibles entre elles,
ce qui peut justifider l'existence de plusieurs (familles de) courbes.
\end{block}
\end{frame}
\subsection{Difficulté intrinsèque du logarithme discret}
\begin{frame}\frametitle{Difficulté du logarithme discret}
\begin{block}{}
Si le logarithme discret est facile, alors
toute implémentation de la courbe sera faible.
\end{block}
\begin{itemize}
\item Il existe des attaques contre des groupes génériques d'ordre $N$,
de complexité $√N$.
\item Pour qu'une courbe soit correcte,
on exige donc qu'il n'existe pas de meilleure attaque.
\end{itemize}
\begin{center}
\includegraphics[width=.3\hsize]{raphael_euclid}
\end{center}
\end{frame}

\begin{frame}\frametitle{Notations}
Dans tout ce qui suit, $E$ désigne la courbe d'équation
\[ y^2 = x^3 + a x + b\]
définie sur le corps $k = \F_p$, et $N = \abs{E(\F_p)}$.
On considère la multiplication $n · P$ pour un scalaire $n$ secret.

On note $\prob(X)$ la probabilité d'un événement~$X$
comprise, selon le contexte,
sur l'ensemble des courbes sur $\F_p$ avec $p$ fixé,
ou sur l'ensemble des courbes sur $\F_p$ avec $p$ de taille fixée.
\end{frame}
\begin{frame}\frametitle{Courbes singulières}
Si $Δ = 4 a^3 + 27 b^2 = 0$, alors
la courbe d'équation $y^2 = x^3 + a x + b$ n'est pas une courbe elliptique :
c'est une cubique singulière.

Le groupe des points réguliers est alors isomorphe à
un groupe additif ou multiplicatif,
et le logarithme discret est sous-exponentiel, voire polynomial.

\begin{block}{}
Il est impératif que $Δ ≠ 0$ (ce qui arrive avec $\prob ≈ 1$).
\end{block}
\end{frame}
\begin{frame}\frametitle{Grand sous-groupe premier}
\begin{itemize}
\item Il existe des attaques génériques de complexité~$O(√q)$,
où $q$ est le plus grand diviseur premier de $N$.
\item Une courbe sûre doit donc avoir $q ≈ N$ ; idéalement, $q = N$.
\item Ceci nécessite de calculer $N$,
ce qui est une tâche relativement coûteuse.
\item La probabilité qu'une courbe aléatoire ait un ordre premier
est approximativement la même que celle
qu'un nombre aléatoire de la taille de $p$ soit premier,
soit $\prob ≈ \frac{1}{\log N}$.
\end{itemize}
\begin{block}{}
\begin{itemize}
\item Il est impératif que $N$ ait un grand facteur premier.
\item Calculer $N$ est l'une des étapes coûteuses
de la génération de la courbe.
\end{itemize}
\end{block}
\end{frame}
\begin{frame}\frametitle{Transfert additif ou multiplicatif}
\begin{itemize}
\item Si $N = p$ alors il existe un homomorphisme de groupe calculable
vers le groupe additif $\F_p$,
et donc le logarithme discret est de complexité polynomiale.
\begin{block}{}
Il est impératif que $N ≠ p$.
\end{block}
\item Cette condition exclut les courbes supersingulières.
\bigskip
\item Soit $e$ le \imp{degré de plongement},
c.à.d. le plus petit entier tel que $N$ divise $p^e - 1$ ;
alors il existe un homomorphisme de groupe calculable vers
le groupe multiplicatif de $\F_{p^e}$,
et donc le logarithme discret a une complexité
sous-exponentielle relativement à $p^e$.

\begin{itemize}
\item Solution : si $e ≈ p$ alors cette
complexité est exponentielle relativement à $p$.
\end{itemize}
\end{itemize}
\begin{block}{}
\begin{itemize}
\item Il est impératif que $e$ soit assez grand ($\prob ≈ 1$).
\item
Pour calculer $e$, il faut connaître la factorisation de $q-1$,
ce qui est sous-exponentiel
(c'est asymptotiquement l'étape la plus coûteuse).
\end{itemize}
\end{block}
\end{frame}
\subsection{Résistance des implémentations de la courbe}
\begin{frame}
\includegraphics[width=.8\hsize]{maes_eavesdropping}
\end{frame}
\begin{frame}{Résistance des implémentations de la courbe}
\begin{block}{}
Il existe des courbes pour lesquelles,
bien que le logarithme discret soit difficile,
certaines implémentations ou certains protocoles sont faibles.
\end{block}
Exemple : multiplication par l'algorithme « doublement et addition »
non protégé.
\begin{center}
\includegraphics[width=24em,height=8em]{spa.pdf}
\hskip 1.2em
 \begin{tabular}{|p{1.45em}|p{1.45em}|p{1.45em}|p{1.45em}|p{1.45em}|p{1.45em}|p{1.45em}|p{1.45em}|} 
 \hline
 D&A&D&D& D&A&D&A\\
 \hline
 \multicolumn{2}{|c|}{1} &
 \vrule width 0pt height 2ex
 0 & 0 & \multicolumn{2}{|c|}{1} & \multicolumn{2}{|c|}{1} \\
 \hline
 \end{tabular}
\end{center}
\end{frame}
\begin{frame}\frametitle{Contre-mesures classiques}
\begin{itemize}
\item Contre les attaques simples :
élimination des branches conditionnelles sur un élément secret.
\begin{itemize}
\item double and always add :
\[ \begin{split}
&Q ← P;\\
&\texttt{for $i = ℓ-2,…,0$:}\\
&\qquad Q_0 ← 2 Q;\; Q_1 ← Q_0 + P;\; Q ← Q_{n_i};
\end{split} \]
\item échelle de Montgomery:
\[\begin{split}
&Q_0 ← P; Q_1 ← 2P;\\
&\texttt{for $i = ℓ-2,…,0$:}\\
&\qquad Q_{1-n_i} ← Q_0 + Q_1;\; Q_{n_i} ← 2 Q_{n_i};
\end{split}\]
\end{itemize}
\item Contre les attaques différentielles :
ne pas manipuler plusieurs fois le même élément secret.
\begin{itemize}
\item masquage aléatoire du secret ($n ← n + r · N$ avec $r$ aléatoire) ;
\item masquage aléatoire de la courbe ($a ← r^4 a, b ← r^6 b$) ;
\item masquage aléatoire du point ($(x:y:1) ← (rx:ry:r)$)...
\end{itemize}
\end{itemize}
\end{frame}
\begin{frame}\frametitle{Petit sous-groupe}
Si la courbe possède un petit sous-groupe, alors
il est possible, dans certains protocoles, d'obtenir de l'information
sur des données secrètes\bib{Lim-Lee 97}.
\begin{itemize}
\item Si le point de base $G$ est d'ordre $m$,
alors l'observation de $a G$ permet de retrouver la valeur $a \pmod{m}$.
\item Suppose un adversaire capable de fournir un point de base
de son choix.
\end{itemize}
\begin{block}{}
Il est préférable que le groupe des points de la courbe
soit d'ordre premier\\ ($\prob = 1$ si $N$ premier).
\end{block}
\end{frame}
\begin{frame}\frametitle{Sécurité de la courbe tordue}
La \imp{courbe tordue} de $E$ est la courbe~$E'$ d'équation
\[ d y^2 = x^3 + a x + b, \qquad
  \text{où $d$ n'est pas un carré dans~$\F_p$.} \]
Pour tout~$x ∈ \F_p$ tel que~$x^3+ax+b ≠ 0$,
il existe un point d'abscisse~$x$ sur
exactement une des deux courbes $E$ et~$E'$.

Dans certains cas, un adversaire peut injecter une abscisse appartenant
à la courbe tordue\bib{Fouque-Lercier-Réal-Vallette 2008} :
\begin{itemize}
\item protocole mal conçu (compression de point),
\item attaque par injection de faute...
\end{itemize}

\begin{block}{}
Il est préférable que la courbe tordue soit sûre.\\
($\prob ≈ \strong{$\frac{1}{\log p}$}$.)
\end{block}
\end{frame}
\begin{frame}\frametitle{Points spéciaux}
Un \imp{point spécial} est un point de la courbe
de la forme $(0, y)$ ou $(x, 0)$.

En présence de tels points, certaines implémentations
peuvent faire fuir de l'information\bib{Goubin 2003}.

\begin{itemize}
\item Des points spéciaux de la forme $(0, y)$ existent
si $b$ est un carré dans~$k$\\ ($\prob = 1/2$).
\item Des points spéciaux de la forme $(x, 0)$ existent
si $N$ est pair\\ ($\prob = 1$ si $N$ premier).
\end{itemize}


\begin{block}{}
Il est préférable que la courbe ne contienne pas de tels points spéciaux.
\end{block}
\end{frame}
\begin{frame}\frametitle{Corps de base spéciaux}
De nombreuses courbes standard ont des corps de base spéciaux :
\[\begin{split} p_{192} &= 2^{192} - 2^{64} - 1\\
&= \texttt{\small 0xfffffffffffffffffffffffffffffffeffffffffffffffff}.
\end{split}\]
\begin{itemize}
\item Puisque $N = p + O(√p)$,
les premiers bits de~$N$ sont les premiers bits de~$p$.
\item Les premiers bits de $n + r · N$
sont les premiers bits de~$r$.
\item Ceci rend la protection par masquage de~$n$
insuffisante\bib{DK2005, BPSY2014, FRV2014...}.
\end{itemize}

\begin{block}{}
Il est préférable que le corps de base ne soit pas d'une forme
spéciale.
\end{block}
\end{frame}
\begin{frame}\frametitle{Loi de groupe unifiée}
Certaines courbes admettent une loi d'addition \imp{unifiée} :
il existe un système de coordonnées permettant d'effectuer
les opérations $P + Q$, $2 P$, $P + 0$ par les mêmes formules.
\begin{itemize}
\item Courbes d'Edwards : $x^2 + y^2 = c^2 (t^2 + d z^2)$, $xy = zt$;
\item Courbes de Jacobi : $y^2 = z^2 + 2 a x^2 + b t^2$, $x^2 = zt$...
\end{itemize}

Ceci est possible sous certaines conditions sur la courbe :
\begin{itemize}
\item Edwards : point de $4$-torsion, $\prob = 17/48$;
\item Jacobi : point de $2$-torsion, $\prob = 2/3$...
\end{itemize}

Ceci ajoute une couche de protection contre les attaques simples.
\end{frame}
\subsection{« Généricité » de la courbe}
\begin{frame}\frametitle{Résistance à de possibles attaques futures}
\begin{itemize}
\item Et s'il existait des familles faibles ?
\item Éviter de produire des courbes trop « particulières ».
\item Vérifier des propriétés satisfaites avec~$\prob ≈ 1$.
\item En particulier, vérifier que différents nombres attachés à la
courbe sont « assez grands ».
\end{itemize}
\begin{center}
\includegraphics[width=.3\hsize]{waterhouse_crystal}
\end{center}
\end{frame}
\begin{frame}\frametitle{Discriminant du corps des endomorphismes}
\begin{itemize}
\item Le \imp{corps des endomorphismes} de~$E$ est
l'extension quadratique~$K$ de~$ℚ$ engendrée par le Frobenius~$φ$.
\item Le polynôme minimal de~$φ$ est~$φ^2 - t φ + p$;
son discriminant est~$D_{φ} = t^2 - 4p < 0$.
\item Le discriminant de~$K$ est donné par $D_{φ} = D_{K} f_{φ}^2$,
où $D_{K}$ ou $D_{K}/4$ est sans facteur carré.
\end{itemize}
\begin{block}{}
En général, $\abs{D_K} ≈ p$ ;
en particulier, $\abs{D_{K}} ≥ √p$ avec~$\prob ≈ 1-O(1/√p)$.
\end{block}
\begin{itemize}
\item Cette condition élimine les valeurs~$D_{K} = -3$ et~$-4$,
correspondant aux $j$-invariants $0$ et~$1728$.
\end{itemize}
\end{frame}
\begin{frame}\frametitle{Nombre de classes}
Le nombre de classes~$h_K$ de~$K$ intervient dans divers
algorithmes liés à~$E$ :
\begin{itemize}
\item c'est le degré du polynôme de Hilbert à factoriser pour la
théorie de la multiplication complexe ;
\item c'est le plus petit degré d'une extension $L/ℚ$
sur laquelle $E$ admet un relèvement fidèle.
\end{itemize}
Ce nombre est minoré en fonction de $D_{K}$:
\[ h(K) ≥ C \frac{\sqrt{\abs{D_K}}}{\log \abs{D_K}}. \]
\begin{itemize}
\item[$⇒$] aucune condition supplémentaire sur $h_K$ n'est a priori
nécessaire.
\end{itemize}
\end{frame}
\begin{frame}\frametitle{Friabilité du nombre de classes}
\begin{itemize}
\item Le nombre de classes $h_K$ est l'ordre du groupe de classes de~$K$.
\item Pour éviter de potentielles attaques utilisant la décomposition
de ce groupe en facteurs élémentaires, on souhaite que
$h_K$ ne soit pas friable (= n'ait pas que des petits facteurs premiers).
\item Un nombre aléatoire $n$ est $n^{1/u}$-friable avec
probabilité~$\prob ≈ u^{-u}$.
\item Par conséquent, $h_K$ a une probabilité négligeable d'être
$(\log p)^{O(1)}$-friable.
\begin{block}{}
En général, $h_K$ a au moins un diviseur premier~$≥ (\log p)^{O(1)}$.
\end{block}
\item Calculer (et factoriser) $h_K$ est sous-exponentiel\bib{Biasse
2010}, mais vérifier que $h_K$ n'est pas logarithmiquement friable
est polynomial.
\end{itemize}
\end{frame}
\begin{frame}\frametitle{Friabilité de la courbe tordue}
\begin{itemize}
\item Si la courbe $E$ a pour cardinal $N = p + 1 - t$,
alors sa tordue $E'$ a pour cardinal $N' = p + 1 + t$.
\item Le cardinal $N'$ est $p^{1/u}$-friable
avec probabilité $\prob ≈ u^{-u}$.
\begin{itemize}
\item Par exemple, $N'$ est $p^{1/4}$-friable avec probabilité $1/256$.
\end{itemize}
\begin{block}{}
On s'attend à ce que $N'$ ait au moins un diviseur premier
de taille polynomiale en~$p$.
\end{block}
\item Le choix du seuil $u$ est délicat ; par exemple, pour $p≈
2^{256}$, une probabilité~$2^{-128}$ correspond aux nombres
$727$-friables...
\item Variante plus stricte de cette condition : $N'$ premier (= courbe
tordue sûre).
\end{itemize}
\end{frame}
\begin{frame}\frametitle{Corps de base spécial}

\end{frame}
\begin{frame}\frametitle{Degré de plongement}

\end{frame}
\begin{frame}\frametitle{Structure multiplicative du corps de base}

\end{frame}
\subsection{Facilitation de l'implémentation}
\begin{frame}\frametitle{Facilitation de l'implémentation}
Certains choix de courbes permettent de disposer d'implémentations
plus rapides ou plus commodes.
\begin{center}
\includegraphics[width=.7\hsize]{turner_speed}
\end{center}
\end{frame}
\begin{frame}\frametitle{Coordonnées jacobiennes rapides}
\end{frame}
\sectionframe{Transparence}
\begin{frame}\frametitle{Transparence}
\begin{center}
\includegraphics[width=.6\hsize]{pompeii_transparent}
\end{center}
\end{frame}


\end{document}
