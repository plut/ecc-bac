\documentclass[francais]{beamer}
\usetheme{anssi}
\usepackage[utf8]{inputenc}
\usepackage[T1]{fontenc}
\usepackage[francais]{babel}
\usepackage{unicode}
\DeclareUnicodeCharacter{2713}{\checkmark}
\usepackage{animate}
\usepackage{bookmark}
\usepackage{graphicx}
\graphicspath{{graphics/}}

\newcommand{\abs}[1]{\left| #1 \right|}
\newcommand{\F}{\mathbb{F}}
\newcommand{\prob}{\mathcal{P}}
\newcommand{\bib}[1]{{\usebeamercolor{emph}\textcolor{fg}{~[#1]}}}

\begin{document}
\title[Quelle courbe elliptique ?]{Quelle courbe elliptique pour la cryptographie ?}
\author[J.-P. Flori]{Jean-Pierre Flori, Jérôme Plût}
\institute[ANSSI]{ANSSI/SDE/ST/LCR}
\date[20/01/2017]{Vendredi 20 janvier 2017}

\begin{frame}<handout:0> \titlepage
\end{frame}

\section{Introduction}

\begin{frame}\frametitle{Pourquoi les courbes elliptiques ?}
\begin{itemize}
\item Un outil classique en cryptographie est
l'échange de clés de Diffie-Hellman, qui repose sur la relation
\[ (g^a)^b = g^{ab} = (g^b)^a, \]
valable dans le groupe multiplicatif des entiers modulo $p$.
\item La sécurité repose sur le problème du \emph{logarithme discret}:
étant donnés $g$ et~$g^a$, il est difficile de calculer $a$.
\item Ce groupe multiplicatif est cependant vulnérable
à certaines attaques spécifiques (GNFS, SNFS)
qui le rendent sous-optimal.
\item Solution : augmenter la taille de $p$...
\item ... ou alors, remplacer le groupe multiplicatif
par un groupe résistant à ces attaques.
\end{itemize}
\end{frame}

\begin{frame}\frametitle{Les courbes elliptiques en cryptographie}
\begin{itemize}
\item Proposées en 1985 par Koblitz et Miller.
\item Fournissent un groupe abélien fini
où le logarithme discret est \emph{difficile}
(plus que dans les groupes multiplicatifs).
\end{itemize}
\begin{example}
\begin{itemize}
\item Pour un niveau de sécurité de 128~bits : une courbe de 256~bits.
\item Taille équivalente en RSA $≈ 3\,072$ bits.
\end{itemize}
\end{example}
\end{frame}

\begin{frame}\frametitle{Le groupe des points d'une courbe elliptique}
\begin{itemize}
\item Une courbe elliptique est donnée par l'équation
\[ y^2 = x^3 + a x + b \pmod{p}, \]
où $p$ est un nombre premier ($\neq 2,3$) et $a, b$ sont deux paramètres.
\item Les points de la courbe forment un groupe abélien
(noté additivement).
\item Le cardinal $N$ de ce groupe est environ $p$ :
$\abs{N-(p+1)} ≤ 2\sqrt{p}$.
\end{itemize}
\begin{block}{}
En général, le groupe des points d'une courbe elliptique
se comporte comme un « groupe générique » :
le logarithme discret a une complexité exponentielle.
\end{block}
\end{frame}

\begin{frame}\frametitle{Pourquoi standardiser ?}
\begin{block}{}
\emph{En général,} le groupe des points d'une courbe elliptique
se comporte comme un « groupe générique » :
le logarithme discret a une complexité \emph{exponentielle}.
\end{block}
\begin{itemize}
\item Plus précisément, la complexité du logarithme discret
est dominée par $\sqrt{q}$, où $q$ est
le \emph{plus grand diviseur premier} du nombre de points de la courbe.
\item Il est donc possible d'atteindre une sécurité de 128 bits
avec une taille de clé de 256 bits.
\begin{itemize}
\item Solution : avoir un nombre de points (presque) premier.
\end{itemize}
\item Certaines courbes particulières sont plus vulnérables :
le problème du logarithme discret peut être transféré
dans un groupe plus facile.
\begin{itemize}
\item Solution : éviter ces cas particuliers.
\end{itemize}
\end{itemize}
\end{frame}

\begin{frame}\frametitle{Génération d'une courbe (I)}
\begin{block}{}
Ces solutions sont gourmandes en calculs,
il n'est donc pas réaliste d'envisager
fabriquer une nouvelle courbe à la volée pour chaque échange.
\end{block}

\begin{itemize}
\item Miller-Rabin sur 3072 bits : 70 ms;
\item SEA sur 256 bits : 4 s ; sur 384 bits : 30 s ; sur 512 bits : 2 min.
\item Environ $1/2 \log p$ courbes à tester pour obtenir un cardinal premier : 10 minutes pour 256 bits\ldots
\item \ldots et $1/2 (\log p)^2$ pour une courbe avec tordue sûre : 35 heures pour 256 bits!
\item (D'autres paramètres doivent également être vérifiés pour la courbe finale.)
\end{itemize}
\end{frame}

\begin{frame}\frametitle{Génération d'une courbe (II)}
\begin{block}{}
Il est possible de générer plus rapidement une courbe en utilisant la multiplication complexe.
\end{block}

\begin{itemize}
\item Lenstra et Miele propose de trouver des solutions de l'équation de Pell donnant des paramètres premiers.
\item Pour être efficace :
\begin{itemize}
\item se restreindre à des polynômes de classes pour lesquels les racines comlexes sont précalculées et représentées par des radicaux;
\item cribler les solutions de l'équation.
\end{itemize}
\end{itemize}
\begin{block}{}
\begin{itemize}
\item Cardinal premier sur 256 bits : 8 ms ;
\item Courbe avec tordue sûre sur 256 bits : 50 ms.
\end{itemize}
\end{block}
Malheureusement de telles courbes sont très spéciales !
\end{frame}

\begin{frame}\frametitle{Première phase de standardisation}
Les premières courbes standardisées ont été produites
au début des années 2000, c'est-à-dire
une fois la recherche dans le domaine suffisamment avancée :
\begin{itemize}
\item possibilité de produire des courbes cryptographiques
(Schoof, Elkies, Atkin) ;
\item identification des classes de courbes faibles.
\end{itemize}
\begin{block}{}
À notre connaissance, ces courbes sont toujours sûres.
\end{block}
\end{frame}

\begin{frame}\frametitle{Courbes standardisées}
\begin{center}\begin{tableau}{lcll}
\entete Année & &Courbes & Tailles \\
2000 & \includegraphics[height=1em]{nsa} &NIST & 192, 224, 256, 384, 521\\
2005 & & Brainpool & 160, 192, 224, 256, 320, 384, 512\\
2010 & \includegraphics[width=1em]{oscca} &OSCCA & 256 \\
2011 & \includegraphics[width=1em]{anssi} &ANSSI & 256 \\
\end{tableau}\end{center}
\end{frame}

\begin{frame}\frametitle{Deuxième phase de standardisation}
Mais de nouvelles préoccupations sont apparues depuis :
\begin{itemize}
\item doutes sur le processus de génération
(possibilité de publier une courbe secrètement vulnérable ?) ;
\item émergence des attaques latérales
(\emph{« side-channel attacks »}) ;
\item progrès scientifiques dans des domaines proches
(logarithme discret multiplicatif\ldots) ;
\item performances sous-optimales (courbes d'Edwards).
\end{itemize}
\begin{block}{Juin 2015}
Le NIST organise un atelier sur le thème de
la standardisation des courbes elliptiques.
\end{block}
\end{frame}

\begin{frame}\frametitle{De nouvelles propositions académiques}
\begin{center}\begin{tableau}{lcll}
\entete Année & &Courbes & Tailles \\
2006 & Bernstein & Curve25519 & 255\\
2014& NXP, Microsoft Research & NUMS & 256, 384, 512 \\
2014 & Bernstein, Lange & Curve41417 & 415 \\
2015 & Hamburg & Ed448-Goldilocks & 448 \\
\end{tableau}\end{center}
\end{frame}

\begin{frame}\frametitle{Courbes utilisées}
\begin{itemize}
\item En pratique, on trouve surtout dans la nature les courbes NIST.
\item Parfois les courbes Brainpool.
\item Et de plus en plus la courbe de Dan Bernstein :
\begin{itemize}
\item OpenSSL,
\item OpenSSH,
\item GnuPG,
\item Signal Protocol\ldots
\end{itemize}
\end{itemize}
\end{frame}

\begin{frame}\frametitle{Quelques restrictions}
Les sujets suivants ne seront pas abordés :
\begin{itemize}
\item Les courbes définies sur des extensions de corps :
\begin{itemize}
\item attaque par restriction de Weil ;
\item attaque par décomposition ;
\item transfert du logarithme discret vers une courbe isogène faible.
\end{itemize}
\item Et en particulier les courbes binaires :
\begin{itemize}
\item D'autant plus que le logarithme discret est quasi-polynomial dans les corps de petite caractéristique.
\end{itemize}
\item Les courbes pour les couplages.
\end{itemize}
Toutes les courbes étudiées ici seront définies sur un corps premier et ordinaires.
\end{frame}

\section{Sécurité}

\begin{frame}\frametitle{Aspects de la sécurité d'une courbe}
Qu'est-ce que qu'une « bonne » courbe pour la cryptographie ?
\begin{itemize}
\item Le problème du logarithme discret est difficile.
\item Les implémentations de la courbes sont résistantes
(par exemple aux attaques par canaux auxiliaires).
\item La courbe ne présente aucun signe particulier suspect.
\item Il est possible de réaliser des implémentations optimisées.
\item La courbe possède des propriétés particulières intéressantes (couplages).
\end{itemize}
\begin{block}{Conditions incompatibles}
Certaines de ces conditions sont incompatibles entre elles,
ce qui peut justifier l'existence de plusieurs (familles de) courbes.
\end{block}
\end{frame}

\begin{frame}\frametitle{Notations}
\begin{itemize}
\item Dans tout ce qui suit, $E$ désigne la courbe d'équation
\[ y^2 = x^3 + a x + b\]
définie sur le corps $k = \F_p$, et $N = \abs{E(\F_p)}$.
On considère la multiplication $n · P$ pour un scalaire $n$ secret.
\bigskip
\item On note $\prob(X)$ la probabilité d'un événement~$X$
comprise, selon le contexte,
sur l'ensemble des courbes sur $\F_p$ avec $p$ fixé,
ou sur l'ensemble des courbes sur $\F_p$ avec $p$ de taille fixée.
\end{itemize}
\end{frame}

\subsection{Difficulté intrinsèque du logarithme discret}


\begin{frame}\frametitle{Difficulté du logarithme discret}
\begin{block}{}
Si le logarithme discret est facile, alors
toute implémentation de la courbe sera faible.
\end{block}
\begin{itemize}
\item Il existe des attaques contre des groupes génériques d'ordre $N$,
de complexité $\sqrt{N}$.
\item Pour qu'une courbe soit correcte,
on exige donc qu'il n'existe pas de meilleure attaque.
\end{itemize}
\end{frame}

\begin{frame}\frametitle{Courbes singulières}

Si $Δ = 4 a^3 + 27 b^2 = 0$, alors
la courbe d'équation $y^2 = x^3 + a x + b$ n'est pas une courbe elliptique :
c'est une cubique singulière.

Le groupe des points réguliers est alors isomorphe à
un groupe additif ou multiplicatif,
et le logarithme discret est sous-exponentiel, voire polynomial.

\begin{block}{}
Il est impératif que $Δ ≠ 0$ (ce qui arrive avec $\prob ≈ 1$).
\end{block}
\end{frame}

\begin{frame}\frametitle{Grand sous-groupe premier}
\begin{itemize}
\item Il existe des attaques génériques de complexité~$O(√q)$,
où $q$ est le plus grand diviseur premier de $N$.
\item Une courbe sûre doit donc avoir $q ≈ N$ ; idéalement, $q = N$.
\item Ceci nécessite de calculer $N$,
ce qui est une tâche relativement coûteuse.
\item La probabilité qu'une courbe aléatoire ait un ordre premier
est approximativement la même que celle
qu'un nombre aléatoire de la taille de $p$ soit premier,
soit $\prob ≈ \frac{1}{\log p}$.
\end{itemize}
\begin{block}{}
\begin{itemize}
\item Il est impératif que $N$ ait un grand facteur premier.
\item Calculer $N$ est l'une des étapes coûteuses
de la génération de la courbe.
\end{itemize}
\end{block}
\end{frame}

\begin{frame}\frametitle{Transfert additif}
\begin{itemize}
\item Si $N = p$ alors il existe un homomorphisme de groupe calculable
vers le groupe additif $\F_p$.
\item Le logarithme discret est alors de complexité polynomiale.
\begin{block}{}
Il est impératif que $N ≠ p$ ($\prob = 1$).
\end{block}
\end{itemize}
\end{frame}

\begin{frame}\frametitle{Transfert multiplicatif}
\begin{itemize}
\item Soit $e$ le \emph{degré de plongement},
c.-à-d. le plus petit entier tel que $N$ divise $p^e - 1$ ;
alors il existe un homomorphisme de groupe calculable vers
le groupe multiplicatif de $\F_{p^e}$,
et donc le logarithme discret a une complexité
sous-exponentielle relativement à $p^e$.
\item Solution : si $e ≈ p$ alors cette
complexité est exponentielle relativement à $p$.
\begin{block}{}
\begin{itemize}
\item Il est impératif que $e$ soit assez grand ($\prob ≈ 1$).
\item
Pour calculer $e$, il faut connaître la factorisation de $q-1$,
ce qui est sous-exponentiel
(c'est asymptotiquement l'étape la plus coûteuse).
\item Il peut être suffisant de montrer que $e$ est assez grand
sans le calculer explicitement.
\end{itemize}
\end{block}
\item (Cette condition exclut les courbes supersingulières.)
\end{itemize}
\end{frame}

\subsection{Résistance des implémentations de la courbe}
\begin{frame}{Résistance des implémentations de la courbe}
\begin{block}{}
Il existe des courbes pour lesquelles,
bien que le logarithme discret soit difficile,
certaines implémentations ou certains protocoles sont faibles.
\end{block}
Exemple : multiplication par l'algorithme « doublement et addition »
non protégé.
\begin{center}
\hspace*{.5em}\includegraphics[width=24em,height=8em]{spa.pdf}

 \begin{tabular}{*{8}{|p{1.7em}}|} 
 \hline
 D&A&D&D& D&A&D&A\\
 \hline
 \multicolumn{2}{|c|}{1} &
 \vrule width 0pt height 2ex
 0 & 0 & \multicolumn{2}{|c|}{1} & \multicolumn{2}{|c|}{1} \\
 \hline
 \end{tabular}
\end{center}
\end{frame}
\begin{frame}\frametitle{Contre-mesures classiques (I)}
\begin{itemize}
\item Contre les attaques simples :
élimination des branches conditionnelles sur un élément secret.
\begin{itemize}
\item double and always add :
\[ \begin{split}
&Q ← P;\\
&\texttt{for $i = ℓ-2,…,0$:}\\
&\qquad Q_0 ← 2 Q;\; Q_1 ← Q_0 + P;\; Q ← Q_{n_i};
\end{split} \]
\item échelle de Montgomery:
\[\begin{split}
&Q_0 ← P; Q_1 ← 2P;\\
&\texttt{for $i = ℓ-2,…,0$:}\\
&\qquad Q_{1-n_i} ← Q_0 + Q_1;\; Q_{n_i} ← 2 Q_{n_i};
\end{split}\]
\end{itemize}
\end{itemize}
\end{frame}

\begin{frame}\frametitle{Contre-mesures classiques (II)}
\begin{itemize}
\item Contre les attaques différentielles :
ne pas manipuler plusieurs fois le même élément secret.
\begin{itemize}
\item masquage aléatoire du secret :
\begin{itemize}
\item $n ← n + r · N$ avec $r$ aléatoire ;
\item $n = \lceil n / r \rceil + n \pmod{r}$ avec $r$ aléatoire ;
\end{itemize}
\item masquage aléatoire de la courbe ($a ← r^4 a, b ← r^6 b$) ;
\item masquage aléatoire du point ($(x:y:1) ← (rx:ry:r)$)\ldots
\end{itemize}
\end{itemize}
\end{frame}

\begin{frame}\frametitle{Corps de base spéciaux}
De nombreuses courbes standardisées ont des corps de base spéciaux :
\[\begin{split} p_{192} &= 2^{192} - 2^{64} - 1\\
&= \texttt{\small 0xfffffffffffffffffffffffffffffffeffffffffffffffff}.
\end{split}\]
\begin{itemize}
\item Puisque $N = p + O(√p)$,
les premiers bits de~$N$ sont les premiers bits de~$p$.
\item Les premiers bits de $n + r · N$
sont les premiers bits de~$r$.
\item Ceci rend la protection par masquage de~$n$
insuffisante\bib{DK2005, BPSY2014, FRV2014, SW2015...}.
\end{itemize}

\begin{block}{}
Il est préférable que le corps de base ne soit pas d'une forme
spéciale.
\end{block}
\end{frame}

\begin{frame}\frametitle{Loi de groupe unifiée (I)}
Certaines courbes admettent une loi d'addition \emph{unifiée} et efficace :
il existe un système de coordonnées permettant d'effectuer
les opérations $P + Q$, $2 P$, $P + 0$ par les mêmes formules.
\begin{itemize}
\item Courbes d'Edwards : $x^2 + y^2 = c^2 (t^2 + d z^2)$, $xy = zt$;
\item Courbes de Jacobi : $y^2 = z^2 + 2 a x^2 + b t^2$, $x^2 = zt$...
\end{itemize}

Ceci est possible sous certaines conditions sur la courbe :
\begin{itemize}
\item Edwards : point de $4$-torsion, $\prob = 17/48$;
\item Jacobi : point de $2$-torsion, $\prob = 2/3$...
\end{itemize}

Ceci ajoute une couche de protection contre les attaques simples.
\end{frame}

\begin{frame}\frametitle{Loi de groupe unifiée (II)}
\begin{itemize}
\item En réalité, il est possible de développer de telles lois d'addition \emph{unifiée} pour toute forme de courbe en s'imposant certaines contraintes.
\item Le problème est de les rendre efficaces.
\end{itemize}
\begin{block}{}
Renes et al. ont proposé une loi d'addition unifiée relativement efficace
pour calculer dans un sous-groupe de cardinal impair
de courbes sous forme de Weierstrass.
\end{block}
\end{frame}

\begin{frame}\frametitle{Points spéciaux}
Un \emph{point spécial} est un point de la courbe
de la forme $(0, y)$ ou $(x, 0)$.

En présence de tels points, certaines implémentations
peuvent faire fuir de l'information\bib{Goubin 2003}.

\begin{itemize}
\item Des points spéciaux de la forme $(0, y)$ existent
si $b$ est un carré dans~$k$ ($\prob = 1/2$).
\item Des points spéciaux de la forme $(x, 0)$ existent
si $N$ est pair ($\prob = 1$ si $N$ premier).
\end{itemize}

\begin{block}{}
Il est préférable que la courbe ne contienne pas de tels points spéciaux.
\end{block}
\begin{itemize}
\item (Des points spéciaux de la forme $(0, y)$ existent toujours sur la courbe
ou sa tordue ($\prob = 1$).)
\end{itemize}
\end{frame}

\begin{frame}\frametitle{Petit sous-groupe}
Si la courbe possède un petit sous-groupe, alors
il est possible d'obtenir de l'information
sur des données secrètes\bib{Lim-Lee 97}.
\begin{itemize}
\item Si le point de base $G$ est d'ordre $m$ avec $m$ petit,
alors l'observation de $a G$ permet de retrouver la valeur $a \pmod{m}$.
\item Suppose un adversaire capable de fournir un point de base
de son choix.
\end{itemize}
\begin{block}{}
Il est préférable que le groupe des points de la courbe
soit d'ordre premier ($\prob = 1$ si $N$ premier).
\end{block}
\end{frame}
\begin{frame}\frametitle{Sécurité de la courbe tordue}
La \emph{courbe tordue} de $E$ est la courbe~$E'$ d'équation
\[ d y^2 = x^3 + a x + b, \qquad
  \text{où $d$ n'est pas un carré dans~$\F_p$.} \]
Pour tout~$x ∈ \F_p$ tel que~$x^3+ax+b ≠ 0$,
il existe un point d'abscisse~$x$ sur
exactement une des deux courbes $E$ et~$E'$.

Dans certains cas, un adversaire peut injecter une abscisse appartenant
à la courbe tordue\bib{Fouque-Lercier-Réal-Valette 2008} :
\begin{itemize}
\item protocole mal conçu (compression de point),
\item manque de tests dans l'implémentation,
\item attaque par injection de faute...
\end{itemize}

\begin{block}{}
Il est préférable que la courbe tordue soit sûre
($\prob ≈ \frac{1}{\log p}$).
\end{block}
\end{frame}


\subsection{« Généricité » de la courbe}

\begin{frame}\frametitle{Résistance à de possibles attaques futures}
\begin{itemize}
\item Et s'il existait des familles faibles ?
\item Éviter de produire des courbes trop « particulières ».
\item Vérifier des propriétés satisfaites avec~$\prob ≈ 1$.
\item En particulier, vérifier que différents nombres attachés à la
courbe sont « assez grands » ou « peu friables ».
\item Plus précisément, $\prob ≈ 1 - 1/\sqrt{p}$
pour que produire une telle courbe
soit aussi difficile qu'attaquer le logarithme discret.
\end{itemize}
\end{frame}

\begin{frame}\frametitle{Degré de plongement}
\begin{itemize}
\item Le degré de plongement est l'ordre de $p$
dans le groupe multiplicatif $\F_q^{×}$.
\begin{block}{}
Le degré de plongement est~$≥ p^{1/4}$
avec probabilité~$\prob ≥ 1-1/√p$.
\end{block}
\end{itemize}
\end{frame}

\begin{frame}\frametitle{Friabilité de la courbe tordue}
\begin{itemize}
\item Si la courbe $E$ a pour cardinal $N = p + 1 - t$,
alors sa tordue $E'$ a pour cardinal $N' = p + 1 + t$.
\bigskip
\item Fixer une probabilité~$1/\sqrt{p}$ correspond aux nombres
$727$-friables pour $p≈2^{256}$\ldots
\item Fixer la borne de friablité est plus satisfaisant:
par exemple, $N'$ est $p^{1/4}$-friable avec probabilité $1/256$
pour $p≈2^{256}$.
\bigskip
\item Variante plus stricte de cette condition : $N'$ premier (= courbe
tordue sûre), mais la courbe devient \og spéciale \fg : $\prob ≈ 1/\log p$.
\end{itemize}
\end{frame}

\begin{frame}\frametitle{Discriminant du corps des endomorphismes}
\begin{itemize}
\item Le \emph{corps des endomorphismes} de~$E$ est
l'extension quadratique~$K$ de~$ℚ$ engendrée par le Frobenius~$φ$.
\item L'\emph{anneaux des endomorphismes} $\mathcal{O}$ de~$E$ est un ordre dans $K$.
\item Le polynôme minimal de~$φ$ est~$φ^2 - t φ + p$;
son discriminant est~$D_{φ} = t^2 - 4p < 0$.
\item Le discriminant de~$K$ est donné par $D_{φ} = D_{K} f_{φ}^2$,
où $D_{K}$ ou $D_{K}/4$ est sans facteur carré et $D_K | D_{\mathcal{O}} | D_{φ}$.
\end{itemize}
\begin{block}{}
En général, $\abs{D_K} ≈ p$ ;
en particulier, $\abs{D_{K}} ≥ √p$ avec $\prob ≈ 1-O(1/√p)$.
\end{block}
\end{frame}

\begin{frame}\frametitle{Nombre de classes}
Le nombre de classes~$h_K$ de~$K$ intervient dans divers
algorithmes liés à~$E$ :
\begin{itemize}
\item c'est le degré du polynôme de Hilbert à factoriser pour la
théorie de la multiplication complexe ;
\item c'est le plus petit degré d'une extension $L/ℚ$
sur laquelle $E$ (ou une courbe isogène) admet un relèvement fidèle.
\end{itemize}
Ce nombre est minoré en fonction de $D_{K}$:
\[ h_K ≥ C \frac{\sqrt{\abs{D_K}}}{\log \abs{D_K}}. \]
\begin{itemize}
\item[$⇒$] aucune condition supplémentaire sur $h_K$ n'est a priori
nécessaire.
\end{itemize}
\end{frame}

\begin{frame}\frametitle{Friabilité du nombre de classes}
\begin{itemize}
\item Pour éviter de potentielles attaques utilisant la décomposition
du groupe de classes de $K$ en facteurs élémentaires, on souhaite que
$h_K$ ne soit pas friable (= n'ait pas que des petits facteurs premiers).
\bigskip
\item Malheureusement,
calculer (et factoriser) $h_K$ est sous-exponentiel\bib{Biasse
2010}\ldots
\item Mais vérifier que $h_K$ n'est pas logarithmiquement friable
est polynomial.
\end{itemize}
\end{frame}

\begin{frame}\frametitle{Structure multiplicative du corps de base}
\begin{itemize}
\item La structure multiplicative de $\F_p^{×}$ dépend
de la factorisation de~$p-1$.
\item Même si elle n'est pas liée au logarithme discret sur une courbe,
en général $p-1$ a au moins un grand diviseur premier.
\begin{block}{}
$p-1$ a un diviseur premier $≥ (\log p)^2$
avec probabilité $≥1-1/√p$.
\end{block}
\end{itemize}
\end{frame}

\subsection{Facilitation de l'implémentation}

\begin{frame}\frametitle{Facilitation de l'implémentation}
Certains choix de courbes permettent de disposer d'implémentations
plus rapides ou plus commodes.
\end{frame}

\begin{frame}\frametitle{Coordonnées jacobiennes rapides}
Une courbe elliptique de la forme $y^2 = x^3 - 3 x + b$
(c'est-à-dire telle que $a=-3$)
permet d'économiser 2 des 10 multiplications requises
pour une addition de points.

\bigskip
Une courbe aléatoire sur $\mathbb F_p$ est isomorphe avec une courbe
$a=-3$ avec probabilité
\begin{itemize}
\item $\prob = 1/4$ si $p ≡ +1 \pmod{4}$,
\item $\prob = 1/2$ si $p ≡ -1 \pmod{4}$.
\end{itemize}
\end{frame}


\begin{frame}\frametitle{Loi de groupe rapide}

Certains modèles de courbes ont des formules d'addition particulièrement rapides :
\begin{itemize}
\item Courbes d'Edwards : $x^2 + y^2 = c^2 (t^2 + d z^2)$, $xy = zt$;
\item Courbes de Jacobi : $y^2 = z^2 + 2 a x^2 + b t^2$, $x^2 = zt$...
\end{itemize}
\bigskip
Ceci est possible sous certaines conditions sur la courbe :
\begin{itemize}
\item Edwards : point de $4$-torsion, $\prob = 17/48$;
\item Jacobi : point de $2$-torsion, $\prob = 2/3$...
\end{itemize}
\end{frame}

\begin{frame}\frametitle{Nombre de points}
Si la courbe a $N < p$ points,
alors un élément de $ℤ/Nℤ$ peut être représenté par un nombre
de la même taille que $p$.

\bigskip
Exactement la moitié des courbes satisfont cette condition, $\prob = 1/2$.
\end{frame}

\begin{frame}\frametitle{Racines carrées rapides}
Si $p ≡ -1 \pmod{4}$ (ou $p ≡ 5 \pmod{8}$) alors
calculer des racines carrées modulo~$p$ est plus facile,
ce qui permet de simplifier la compression de points.
\end{frame}

\begin{frame}\frametitle{Arithmétique dédiée}
\begin{itemize}
\item Certains choix de corps de base,
comme les nombres premiers « spéciaux » des courbes NIST,
permettent une arithmétique plus rapide.
\bigskip
\item De même, choisir des petites valeurs pour les paramètres $a$ et $b$
permet d'accélérer l'arithmétique de la courbe :
c'est le cas de toutes les propositions académiques récentes.
\end{itemize}
\end{frame}

\subsection{Diversité}
\begin{frame}\frametitle {Résumé (I)}
\begin{center}\begin{tableau}{*{5}{l}}
\entete       & NIST & Brainpool & ANSSI & OSCCA\\
$N$ premier   &  ✓   & ✓         & ✓     & ✓   \\
$p$ ordinaire &      & ✓         & ✓     & ✓   \\
Loi unifiée   &  ✓   & ✓         & ✓     & ✓    \\
Tordue sûre   &      &           &       &      \\
Générique & &✓&✓&✓ \\
\end{tableau}\end{center}
\end{frame}

\begin{frame}\frametitle {Résumé (II)}
\begin{center}\begin{tableau}{*{4}{l}}
\entete       & NUMS & Curve25519/41417 & Ed448-Goldilocks\\
$N$ premier   & & &\\
$p$ ordinaire & & &\\
Loi unifiée   &✓ &✓ &✓\\
Tordue sûre   &✓ &✓ &✓\\
Générique  & & &\\
\end{tableau}\end{center}
\end{frame}

\begin{frame}\frametitle{Différents critères pour différents usages}
\begin{itemize}
\item Les critères précédemment énoncés ne peuvent être tous vérifiés.
\item En particulier, compromis vitesse/généricité.
\item Mais aussi, facilité d'implémentation/protection latérale.
\end{itemize}
\begin{block}{}
Il est souhaitable d'utiliser (et de normaliser) différentes courbes
pour différents usages !
\end{block}
\end{frame}

\section{Transparence}

\begin{frame}\frametitle{Transparence}
Afin de lever tout soupçon de manipulation, on cherche à définir
un programme \emph{déterministe} pour échantillonner une courbe.
\begin{itemize}
\item Processus de génération entièrement reproductible.
\item Peut être un générateur pseudo-aléatoire (généricité)
ou une énumération des petites valeurs (efficacité).
\item Possibilité de vérifier facilement qu'une courbe est issue
d'un tel processus sans le rejouer intégralement.
\end{itemize}
\bigskip
\begin{itemize}
\item Parmi les conditions de sécurité, seules quelques-unes
influenceront probablement le résultat final :
\begin{itemize}
\item sécurité ou non de la courbe tordue,
\item et peut-être le choix des bornes de friabilité.
\end{itemize}
\end{itemize}
\end{frame}

\subsection{Le processus de génération}

\begin{frame}\frametitle{Générateur pseudo-aléatoire}
Un schéma classique de génération de courbe procède à la façon d'un générateur pseudo-aléatoire :
\begin{itemize}
\item Initialiser un état courant à l'aide d'une graine \og aléatoire \fg.
\item Tant qu'une courbe correcte n'a pas été échantillonnée :
\begin{itemize}
\item Mettre à jour l'état courant et échantillonner de nouveaux coefficients à l'aide d'une fonction de hachage.
\item Tester la courbe.
\end{itemize}
\item (Changer de nombre premier à chaque courbe peut aider pour obtenir une tordue sûre.)
\end{itemize}
\end{frame}

\begin{frame}\frametitle{Brainpool (I)}

\begin{itemize}
\item Le processus Brainpool utilise SHA-1 pour la mise à jour et l'état interne.

\bigskip

\item Les critères de sécurité sont:
\begin{enumerate}
\item $N$ premier;
\item $N \neq p$;
\item $(N-1)/e < 100$;
\item $h_K > 10^7$.
\end{enumerate}

\bigskip

\item (Le processus NIST est relativement similaire.)
\end{itemize}
\end{frame}

\begin{frame}\frametitle{Brainpool (II)}
Un nombre premier pseudo-aléatoire est généré:
\begin{enumerate}
\item Initialiser l'état interne $s_p$ avec une  graine de 160 bits.
\item Échantillonner un nombre $c$ et trouver le plus petit premier $p \equiv 3 \pmod{4}$ plus grand que $c$.
\item Si $p$ n'est pas de la taille exacte voulue, mettre à jour $s_p$ et recommencer.
\end{enumerate}

Enfin une courbe avec des coefficients pseudo-aléatoires
et un point base aléatoire sont générés.
\end{frame}

\begin{frame}\frametitle{Brainpool (III)}
\begin{enumerate}
\item L'état courant $s_{ab}$ est initialisé à l'aide d'une  graine de 160 bits.
\item Échantillonner $A$.
\item Si $-3 \neq A*Z^4 \pmod{p}$, mettre à jour $s_{ab}$ et recommencer.
\item Mettre à jour $s_{ab}$.
\item Échantillonner $B$.
\item Si $B$ est un carré modulo $p$, mettre à jour $s_{ab}$ et recommencer.
\item Si la courbe est singulière, mettre à jour $s_{ab}$ et recommencer.
\item Si la courbe ne vérifie pas les critères de sécurité, mettre à jour $s_{ab}$ et recommencer.
\item Mettre à jour $s_{ab}$.
\item Échantillonner $k$.
\item Calculer $\pm Q$ de plus petite abscisse.
\item Calculer $G = k \cdot \pm Q$.
\end{enumerate}
\end{frame}


\subsection{Graine}

\begin{frame}\frametitle{Choix de la graine (I)}
\begin{itemize}
\item Quand l'utilisation d'une graine est nécessaire, il faut être capable de le justifier : elle ne peut être aléatoire.
\item Habituellement les décimales d'une constante classique sont utilisées ($\pi$, $e$).
\item Ou le haché d'une citation littéraire ou d'une phrase en relation avec l'utilisation de la courbe.
\item Mais le nombre de tels choix possibles est grand\ldots
\end{itemize}
\end{frame}

\begin{frame}\frametitle{Choix de la graine (II)}
\begin{itemize}
\item Pour éviter tout soupçon sur le choix de la graine :
\begin{itemize}
\item utiliser un schéma d'engagement sur des fragments de la graine ;
\item utiliser un résultat hors de portée de manipulation
(cours de la Bourse, résultat sportif, loterie publique, Bitcoin).
\end{itemize}
\item Mais cela n'est pas non plus satisfaisant pour certains...
\end{itemize}
\end{frame}

\begin{frame}\frametitle{Choix de la graine (III)}
\begin{center}
%\includegraphics[width=0.7\hsize]{keyboard_cat}
\animategraphics[loop,autoplay,width=0.7\hsize]{12}{animated_cat/keyboard_cat-}{0}{93}
\end{center}
\end{frame}

\subsection{Manipulation et rigidité}

\begin{frame}\frametitle{Manipulation}
\begin{itemize}
\item Le choix des processus de mise à jour de l'état courant et d'échantillonnage des courbes est relativement vaste.
\item Cette latitude pourrait être utilisée pour s'assurer que la courbe finale tombe dans une classe faible.
\begin{block}{DJB about Brainpool}
\small
Several unexplained decisions: Why SHA-1 instead of, e.g., RIPEMD-160 or SHA-256? Why use 160 bits of hash input independently of the curve size? Why pi and e instead of, e.g., sqrt(2) and sqrt(3)? Why handle separate key sizes by more digits of pi and e instead of hash derivation? Why counter mode instead of, e.g., OFB? Why use overlapping counters for A and B (producing the repeated 26DC5C6CE94A4B44F330B5D9)? Why not derive separate seeds for A and B?
\end{block}
\item Courbes \texttt{BADA55EC}.
\end{itemize}
\end{frame}

\begin{frame}\frametitle{Rigidité}
\begin{itemize}
\item Certains auteurs proposent donc de restreindre ces degrés de libertés en échantillonnant les courbes en fonction des valeurs absolues de leurs coefficients.
\item Plus de choix de fonctions de hachage ou de graine.
\item (Un autre avantage est d'obtenir de petits coefficients pour la courbe finale.)
\item Il reste cependant des choix arbitraires à faire...
\end{itemize}
\end{frame}

\subsection{Certificats de génération}

\begin{frame}\frametitle{Certificats de génération}

Fournir un \emph{certificat} permettant
de vérifier les propriétés sans refaire tous les calculs :
\begin{itemize}
\item nombre de points,
\item discriminant et nombre de classes,
\item degré de plongement.
\end{itemize}
Toutes les étapes du processus sont certifiées,
y compris les courbes rejetées.
\end{frame}

\begin{frame}\frametitle{Nombre de points premier}
\begin{itemize}
\item On connaît la borne de Hasse-Weil: $\abs{N-(p+1)} ≤ 2√p$.
\medskip
\item Si $G ≠ 0$ est tel que $q · G = 0$ avec $q ≥ p-2√p+1$ premier,
alors $q = N$.
\medskip
\item Certificat de primalité de $N$: $(G, q, Π)$,
où $Π$~est une preuve de primalité de~$q$.
\begin{itemize}
\item Pour les tailles usuelles de courbes elliptiques,
$Π$ peut être laissée vide ;
en effet, pour ces tailles, une preuve directe par APR-CL
est probablement plus rapide qu'une preuve par certificat ECPP.
\end{itemize}
\medskip
\item Ce certificat est produit une seule fois, pour la courbe finale.
\item Taille du certificat et vérification en $O(\log^2 p)$ opérations.
\end{itemize}
\end{frame}

\begin{frame}\frametitle{Nombre de points composé (I)}
\begin{itemize}
\item Si $n < 2 (√p-1)^2$ est composé et si $P ≠ 0$, $n · P = 0$,
alors l'ordre de la courbe est composé.
\begin{itemize}
\item preuve: soit $d = \gcd (n, N)$; alors $d ≠ 1$ car~$P ≠ 0$.
Si $d ≠ N$ alors $d$~est un diviseur strict de~$N$.
Si $d = N$ alors $N$~divise~$n$ ; puisque $n/2 < (√p-1)^2$,
on a nécessairement~$N = n$, qui est composé.
\end{itemize}
\item Certificat de composition de~$N$ : $(n, c, P)$,
où $c$~est un témoin de composition (Miller-Rabin) de~$n$.
\item Complexité pour chacune des $O(\log p)$ ou $O(\log^2 p)$ courbes :
\begin{itemize}
\item production du certificat : $O(\log p)$~multiplications;
\item taille du certificat : $O(\log p)$;
\item vérification : $O(\log p)$~multiplications.
\end{itemize}
\end{itemize}
\end{frame}

\begin{frame}\frametitle{Nombre de points composé (II)}
\begin{itemize}
\item En pratique, le calcul du nombre de points $N$
se fait en calculant $N \pmod{ℓ}$ pour de petits nombres premiers~$ℓ$.
\item Si on trouve $N ≡ 0 \pmod{ℓ}$, on peut arrêter le calcul
pour cette courbe !
\bigskip
\item Dans ce cas, on trouve un polynôme $f_{ℓ}$, de degré~$O(ℓ)$,
dont les racines sont les abscisses de points de $ℓ$-torsion.
\item Trouver une racine de ce polynôme (par Cantor-Zassenhaus)
a un coût comparable à celui du calcul de $N \pmod{ℓ}$.
\item Dans ce cas, $(ℓ, P)$ est un certificat de composition.
\end{itemize}
\end{frame}

\begin{frame}\frametitle{Discriminant}
\begin{itemize}
\item Le calcul du discriminant nécessite de factoriser
le nombre~$D_{φ} = t^2 - 4p$.
\item C'est asymptotiquement l'une des étapes dominantes
du processus de génération de la courbe ($L(1/3)$).
\item Certificat : la factorisation de $D_{φ}$ (vérification quadratique).
\end{itemize}
\end{frame}

\begin{frame}\frametitle{Nombre de classes}
\begin{itemize}
\item Le calcul exact du nombre de classes est
coûteux
($L(1/2)$, records à 110 chiffres décimaux).
\item Mais la certification simple (quadratique avec factorisation et générateurs).
\item Pour des grandes tailles de corps de base hors de portée :
\begin{itemize}
\item Il suffit encore d'un élément pour prouver que le nombre de classes est
suffisamment grand et non-friable.
\item Il est impossible de prouver réalistement que la courbe
a été rejetée pour cause de nombre de classes trop petit ou trop friable.
\item Solution : générer quelques éléments (déterministes) du groupe de classes
et prouver que l'on ne peut pas prouver avec ces éléments
que le nombre de classes est assez grand.
\end{itemize}
\end{itemize}
\end{frame}

\begin{frame}\frametitle{Exemple}
\begin{itemize}
\item Fonction d'échantillonnage pour une graine~$s$:
\begin{itemize}
\item $p$ = le plus petit premier $≥ s$;
\item $g$ = le plus petit générateur de $\F_p^{×}$;
\item courbes de la forme $y^2 = x^3 - 3x + b$, $b = g, g^2, …$.
\end{itemize}
\bigskip
\item Conditions :
\begin{itemize}
\item $N$ et $N'$ premiers ;
\item $Δ ≠ 0$, $N, N' ≠ p, p + 1$;
\item degré de plongement de~$E$, $E'$ au moins~$p^{1/4}$;
\item nombre de classes $≥ p^{1/4}$.
\end{itemize}
\end{itemize}
\end{frame}

\begin{frame}\frametitle{Exemple de certificat}
Pour la graine $s = 2015$: $p = 2017$, $g = 5$,

\begin{block}{\small\tt Curve}\small\tt
(2017, -3, 625)\\
order = 2063, point = (0, 25)\\
twist\_order = 1973\\
disc\_factors = \{6043\}\\
class\_number = 9, form = (17,3,89)\\
embedding\_degree = 1031, factors = \{2, 1031\}\\
twist\_embedding\_degree = 493, factors = \{2, 17, 29\}
\end{block}
\vskip -1.7em

\begin{block}{\small\tt Rejected curves}\small\tt
((2017, -3, 5), composite, 2065, witness, 1679,
  point, (1,258))\\
((2017, -3, 25), torsion\_point, 3, point, (448, 288))\\
((2017, -3, 125), torsion\_point, 2, point, (982, 0))
\end{block}
\end{frame}

\begin{frame}[standout]
\Huge \textbf{Questions ?}
\end{frame}
\end{document}
